\documentclass[12pt]{article}\usepackage[]{graphicx}\usepackage[]{color}
%% maxwidth is the original width if it is less than linewidth
%% otherwise use linewidth (to make sure the graphics do not exceed the margin)
\makeatletter
\def\maxwidth{ %
  \ifdim\Gin@nat@width>\linewidth
    \linewidth
  \else
    \Gin@nat@width
  \fi
}
\makeatother

\definecolor{fgcolor}{rgb}{0.345, 0.345, 0.345}
\newcommand{\hlnum}[1]{\textcolor[rgb]{0.686,0.059,0.569}{#1}}%
\newcommand{\hlstr}[1]{\textcolor[rgb]{0.192,0.494,0.8}{#1}}%
\newcommand{\hlcom}[1]{\textcolor[rgb]{0.678,0.584,0.686}{\textit{#1}}}%
\newcommand{\hlopt}[1]{\textcolor[rgb]{0,0,0}{#1}}%
\newcommand{\hlstd}[1]{\textcolor[rgb]{0.345,0.345,0.345}{#1}}%
\newcommand{\hlkwa}[1]{\textcolor[rgb]{0.161,0.373,0.58}{\textbf{#1}}}%
\newcommand{\hlkwb}[1]{\textcolor[rgb]{0.69,0.353,0.396}{#1}}%
\newcommand{\hlkwc}[1]{\textcolor[rgb]{0.333,0.667,0.333}{#1}}%
\newcommand{\hlkwd}[1]{\textcolor[rgb]{0.737,0.353,0.396}{\textbf{#1}}}%

\usepackage{framed}
\makeatletter
\newenvironment{kframe}{%
 \def\at@end@of@kframe{}%
 \ifinner\ifhmode%
  \def\at@end@of@kframe{\end{minipage}}%
  \begin{minipage}{\columnwidth}%
 \fi\fi%
 \def\FrameCommand##1{\hskip\@totalleftmargin \hskip-\fboxsep
 \colorbox{shadecolor}{##1}\hskip-\fboxsep
     % There is no \\@totalrightmargin, so:
     \hskip-\linewidth \hskip-\@totalleftmargin \hskip\columnwidth}%
 \MakeFramed {\advance\hsize-\width
   \@totalleftmargin\z@ \linewidth\hsize
   \@setminipage}}%
 {\par\unskip\endMakeFramed%
 \at@end@of@kframe}
\makeatother

\definecolor{shadecolor}{rgb}{.97, .97, .97}
\definecolor{messagecolor}{rgb}{0, 0, 0}
\definecolor{warningcolor}{rgb}{1, 0, 1}
\definecolor{errorcolor}{rgb}{1, 0, 0}
\newenvironment{knitrout}{}{} % an empty environment to be redefined in TeX

\usepackage{alltt}
\usepackage{amsmath}
\usepackage{bm}
\usepackage{fullpage}
\usepackage{mathrsfs}

\DeclareMathAlphabet{\mathpzc}{OT1}{pzc}{m}{it}

\newcommand{\mat}{\bm}
\newcommand{\trans}{^\top}
\newcommand{\vc}{\mathrm{vec}}
\newcommand{\dnorm}{\mathcal{N}}
\newcommand{\dexpfam}{\mathcal{D}}

\author{Steve Walker}
\title{Making general deviance functions for mixed/factor models in lme4}
\date{}
\IfFileExists{upquote.sty}{\usepackage{upquote}}{}
\begin{document}

\maketitle
\tableofcontents



\section{Introduction}

The generalized linear mixed model (GLMM) in \texttt{lme4} takes the
form,
\begin{equation}
  \label{eq:6}
  \bm\eta = \bm X\bm\beta + \bm Z\bm b
\end{equation}
\begin{equation}
  \label{eq:10}
  \bm b = \bm\Lambda_{\bm\theta}\bm u
\end{equation}
\begin{equation}
  \label{eq:7}
  \bm u \sim \dnorm(0, \bm I)
\end{equation}
\begin{equation}
  \label{eq:8}
  \bm y \sim \dexpfam (\bm\eta, \bm\phi)
\end{equation}
where $\dexpfam$ is an exponential family distribution. This GLMM is
suitable for modelling a wide variety of data.  However, in community
ecology the response variables are often species abundances or species
presence-absence, and such data are often characterized by
correlations between species even after the effects of environmental,
phylogenetic, and space are accounted for.  These correlations among
species are typically due to either unmeasured site and species
characteristics and species interactions.  These correlations can be
accounted for by allowing $\bm Z$ to depend on a vector of parameters,
$\bm\psi$.  Therefore the model that we will be using is given by,
\begin{equation}
  \label{eq:6}
  \bm\eta = \bm X\bm\beta + \bm Z_{\bm\psi}\bm\Lambda_{\bm\theta}\bm u
\end{equation}
We now have three parameter vectors:
\begin{itemize}
\item covariance parameters, $\bm\theta$, \texttt{covar}
\item fixed effect parameters, $\bm\beta$, \texttt{fixef}
\item general factor loadings, $\bm\psi$, \texttt{loads}
\end{itemize}
Here I illustrate a function, \texttt{mkGeneralGlmerDevfun}, in the
\texttt{lme4ord} package, which can be used to fit such models.

\section{Model specification}

\subsection{Factor models}

\subsection{Covariance template models}

Consider an \texttt{lmer} model with the following random effects term,
\begin{equation}
  \label{eq:9}
  \mathtt{(0 + explVar:grpFac1 | grpFac2)}
\end{equation}

\begin{knitrout}
\definecolor{shadecolor}{rgb}{0.969, 0.969, 0.969}\color{fgcolor}\begin{kframe}
\begin{alltt}
\hlstd{form} \hlkwb{<-} \hlstd{y} \hlopt{~} \hlnum{1} \hlopt{+} \hlstd{(}\hlnum{0} \hlopt{+} \hlstd{x}\hlopt{:}\hlstd{g1} \hlopt{|} \hlstd{g2)}
\hlstd{dl} \hlkwb{<-} \hlkwd{data.list}\hlstd{(}\hlkwc{y} \hlstd{=} \hlkwd{matrix}\hlstd{(}\hlkwd{rnorm}\hlstd{(}\hlnum{10} \hlopt{*} \hlnum{1}\hlstd{),} \hlnum{10}\hlstd{,} \hlnum{1}\hlstd{),} \hlkwc{x} \hlstd{=} \hlkwd{rnorm}\hlstd{(}\hlnum{10}\hlstd{),}
                \hlkwc{g1} \hlstd{= letters[}\hlnum{1}\hlopt{:}\hlnum{10}\hlstd{],} \hlkwc{g2} \hlstd{= LETTERS[}\hlnum{1}\hlstd{],}
                \hlkwc{dimids} \hlstd{=} \hlkwd{c}\hlstd{(}\hlstr{"g1"}\hlstd{,} \hlstr{"g2"}\hlstd{))}
\hlstd{df} \hlkwb{<-} \hlkwd{as.data.frame}\hlstd{(dl)}
\hlstd{m} \hlkwb{<-} \hlkwd{lmer}\hlstd{(form, df,} \hlkwc{control} \hlstd{=} \hlkwd{lmerControl}\hlstd{(}
                        \hlkwc{check.nobs.vs.nRE} \hlstd{=} \hlstr{"ignore"}\hlstd{,}
                        \hlkwc{check.nlev.gtr.1} \hlstd{=} \hlstr{"ignore"}\hlstd{))}
\end{alltt}


{\ttfamily\noindent\color{warningcolor}{\#\# Warning in commonArgs(par, fn, control, environment()): maxfun < 10 * length(par)\textasciicircum{}2 is not recommended.}}

{\ttfamily\noindent\color{warningcolor}{\#\# Warning in optwrap(optimizer, devfun, getStart(start, rho\$lower, rho\$pp), : convergence code 1 from bobyqa: bobyqa -- maximum number of function evaluations exceeded}}

{\ttfamily\noindent\color{warningcolor}{\#\# Warning in commonArgs(par, fn, control, environment()): maxfun < 10 * length(par)\textasciicircum{}2 is not recommended.}}

{\ttfamily\noindent\color{warningcolor}{\#\# Warning in optwrap(optimizer, devfun, opt\$par, lower = rho\$lower, control = control, : convergence code 1 from bobyqa: bobyqa -- maximum number of function evaluations exceeded}}

{\ttfamily\noindent\color{warningcolor}{\#\# Warning in checkConv(attr(opt, "{}derivs"{}), opt\$par, ctrl = control\$checkConv, : Model failed to converge with max|grad| = 0.769167 (tol = 0.002, component 1)}}

{\ttfamily\noindent\color{warningcolor}{\#\# Warning in checkConv(attr(opt, "{}derivs"{}), opt\$par, ctrl = control\$checkConv, : Model failed to converge: degenerate\ \ Hessian with 1 negative eigenvalues}}\end{kframe}
\end{knitrout}


%% Suppose that $\bm V = \bm L\bm L^\top$ is a covariance matrix over the
%% levels of a grouping factor (e.g. a phylogenetic covariance over
%% species), where $\bm L$ is a Cholesky factor.  Suppose that this
%% grouping factor is crossed with another grouping factor (e.g. species
%% are crossed with sites).  There are two immediate possible models for
%% the structure of covariance over the site-species combinations: a
%% dense model and a block-diagonal model.  To illustrate them we
%% consider two sites for simplicity.  The dense model assumes that
%% species are phylogenetically correlated both within and among sites,
%% \begin{equation}
%%   \label{eq:2}
%%   \bm\Sigma_\theta = \bm\Lambda_\theta\bm\Lambda_\theta^\top =
%%   \theta^2\begin{bmatrix}
%%     \bm L & \bm 0 \\
%%     \bm L & \bm 0 \\
%%   \end{bmatrix}
%%   \begin{bmatrix}
%%     \bm L^\top & \bm L^\top \\
%%     \bm 0      & \bm 0      \\
%%   \end{bmatrix} = 
%%   \theta^2\begin{bmatrix}
%%     \bm V & \bm V \\
%%     \bm V & \bm V \\
%%   \end{bmatrix}
%% \end{equation}
%% whereas the block-diagonal model assumes phylogenetic correlation only
%% within sites,
%% \begin{equation}
%%   \label{eq:2}
%%   \bm\Sigma_\theta = \bm\Lambda_\theta\bm\Lambda_\theta^\top =
%%   \theta^2\begin{bmatrix}
%%     \bm L & \bm 0 \\
%%     \bm 0 & \bm L \\
%%   \end{bmatrix}
%%   \begin{bmatrix}
%%     \bm L^\top & \bm 0      \\
%%     \bm 0      & \bm L^\top \\
%%   \end{bmatrix} = 
%%   \theta^2\begin{bmatrix}
%%     \bm V & \bm 0 \\
%%     \bm 0 & \bm V \\
%%   \end{bmatrix}
%% \end{equation}
%% Interestingly, given that \texttt{lme4} parameterizes covariance
%% models by Cholesky factors, both models require the same amount of
%% structural zeros in the parameterization (because each requires one
%% $\bm L$ per site) even though the covariance matrices themselves
%% differ strongly in the number structural zeros.  That is, the dense
%% model is only dense on the covariance scale, not on the Cholesky
%% scale.

%% Suppose that an explanatory variable, $\bm x$, is measured at each
%% site, and that it has a phylogenetically correlated effect given by
%% one of the above models.  Let $\bm D$ be the matrix with $\bm x$ on
%% the diagonal and $\bm I$ be an identity matrix of size equal to the
%% number of species.  The random effects model matrix would then
%% be given by the diagonal matrix
%% \begin{equation}
%%   \label{eq:3}
%%   \bm Z = \bm D \otimes \bm I
%% \end{equation}

%% Note that it is always possible to put the Cholesky factors in the
%% model matrix.  In this case we have,
%% \begin{equation}
%%   \label{eq:4}
%%   \bm\Lambda_\theta = \theta\bm I
%% \end{equation}
%% and,
%% \begin{equation}
%%   \label{eq:5}
%%   \bm Z =
%%   (\bm D \otimes \bm I)
%%   \begin{bmatrix}
%%     \bm L & \bm 0 \\
%%     \bm L & \bm 0 \\
%%   \end{bmatrix}
%% \end{equation}
%% or
%% \begin{equation}
%%   \label{eq:5}
%%   \bm Z =
%%   (\bm D \otimes \bm I)
%%   \begin{bmatrix}
%%     \bm L & \bm 0 \\
%%     \bm 0 & \bm L \\
%%   \end{bmatrix}
%% \end{equation}
%% because $\bm Z\bm\Lambda_\theta$ remains unchanged by this
%% redefinition.

%% Note also that this model does not allow for phylogenetic correlations
%% among explanatory variables.

%% \subsubsection{Dense templates}

%% The idea is to let the relative covariance factor be,
%% \begin{equation}
%%   \label{eq:1}
%%     \Lambda_\theta =
%%     \begin{bmatrix}
%%       \theta\bm T & \theta\bm T & \cdots & \theta\bm T \\
%%       \theta\bm T & \theta\bm T & \cdots & \theta\bm T \\
%%       \vdots      & \vdots      & \ddots & \vdots \\
%%       \theta\bm T & \theta\bm T & \cdots & \theta\bm T \\
%%     \end{bmatrix} = 
%%     \theta\begin{bmatrix}
%%       \bm T  & \bm T  & \cdots & \bm T \\
%%       \bm T  & \bm T  & \cdots & \bm T \\
%%       \vdots & \vdots & \ddots & \vdots \\
%%       \bm T  & \bm T  & \cdots & \bm T \\
%%     \end{bmatrix}
%% \end{equation}
%% where $\bm T$ is an \emph{a priori} known triangular matrix
%% (e.g. Cholesky factor of a phylogenetic covariance matrix).

%% To specify this model 

%% \subsubsection{Block diagonal repeating templates}

%% \begin{equation}
%%   \label{eq:1}
%%     \Lambda_\theta =
%%     \theta\begin{bmatrix}
%%       \bm T  & \bm 0  & \cdots & \bm 0 \\
%%       \bm 0  & \bm T  & \cdots & \bm 0 \\
%%       \vdots & \vdots & \ddots & \vdots \\
%%       \bm 0  & \bm 0  & \cdots & \bm T \\
%%     \end{bmatrix}
%% \end{equation}

\subsubsection{Multiparameter template models}

For the future.  The idea is to take a singular value decomposition of
the triangular template, $\bm T = \bm U \bm D \bm V^\top$.  The
simple template model can be expressed as $\theta\bm T = \bm U
(\theta\bm D) \bm V^\top$.  Therefore, the simple template model can be
generalized by letting the singular values be a function of a
parameter vector, $\bm\theta$.

\section{Examples}

\subsection{Simple simulation example}

I simulate a small data set with the following dimensions.
\begin{knitrout}
\definecolor{shadecolor}{rgb}{0.969, 0.969, 0.969}\color{fgcolor}\begin{kframe}
\begin{alltt}
\hlstd{n} \hlkwb{<-} \hlnum{60} \hlcom{# samples }
\hlstd{p} \hlkwb{<-} \hlnum{2}  \hlcom{# fixed effects}
\hlstd{q} \hlkwb{<-} \hlnum{30} \hlcom{# random effects}
\end{alltt}
\end{kframe}
\end{knitrout}
The covariance factor, $\Lambda_{\bm\theta}^\top$, is dense (although
we must specify it with a sparse structure).
\begin{knitrout}
\definecolor{shadecolor}{rgb}{0.969, 0.969, 0.969}\color{fgcolor}\begin{kframe}
\begin{alltt}
\hlstd{covTemplate} \hlkwb{<-} \hlkwd{as}\hlstd{(}\hlkwd{chol}\hlstd{(}\hlkwd{cov}\hlstd{(}\hlkwd{matrix}\hlstd{(}\hlkwd{rnorm}\hlstd{((q}\hlopt{+}\hlnum{1}\hlstd{)}\hlopt{*}\hlstd{q), q} \hlopt{+} \hlnum{1}\hlstd{, q))),}
                  \hlstr{"sparseMatrix"}\hlstd{)}
\hlstd{Lambdat} \hlkwb{<-} \hlstd{covTemplate}
\end{alltt}
\end{kframe}
\end{knitrout}
This structure could represent a phylogenetic covariance matrix for
example.
\begin{knitrout}
\definecolor{shadecolor}{rgb}{0.969, 0.969, 0.969}\color{fgcolor}\begin{kframe}
\begin{alltt}
\hlkwd{image}\hlstd{(Lambdat)}
\end{alltt}
\end{kframe}
\includegraphics[width=\maxwidth]{figure/viewLambdat-1} 

\end{knitrout}
The transposed random effects model matrix and fixed effects model
matrix are given by the following.
\begin{knitrout}
\definecolor{shadecolor}{rgb}{0.969, 0.969, 0.969}\color{fgcolor}\begin{kframe}
\begin{alltt}
\hlstd{Zt} \hlkwb{<-} \hlkwd{sparseMatrix}\hlstd{(}\hlkwc{i} \hlstd{=} \hlkwd{rep}\hlstd{(}\hlnum{1}\hlopt{:}\hlstd{q,} \hlnum{6}\hlstd{),}
                   \hlkwc{j} \hlstd{=} \hlkwd{as.vector}\hlstd{(}\hlkwd{outer}\hlstd{(}\hlkwd{rep}\hlstd{(}\hlnum{1}\hlopt{:}\hlnum{10}\hlstd{,} \hlnum{3}\hlstd{),} \hlkwd{seq}\hlstd{(}\hlnum{0}\hlstd{,} \hlnum{50}\hlstd{,} \hlnum{10}\hlstd{),} \hlstr{"+"}\hlstd{)),}
                   \hlkwc{x} \hlstd{=} \hlkwd{rep}\hlstd{(}\hlkwd{rnorm}\hlstd{(}\hlnum{10}\hlstd{),} \hlnum{18}\hlstd{))}
\hlstd{X} \hlkwb{<-} \hlkwd{matrix}\hlstd{(}\hlkwd{rnorm}\hlstd{(n} \hlopt{*} \hlstd{p), n, p)}
\end{alltt}
\end{kframe}
\end{knitrout}
The matrix has the following pattern.  I don't know what this might
represent, but I just wanted to show that essentially any structure
will be fine.  For real examples, one may freely compute Kronecker
products and prewhiten, etc.
\begin{knitrout}
\definecolor{shadecolor}{rgb}{0.969, 0.969, 0.969}\color{fgcolor}\begin{kframe}
\begin{alltt}
\hlkwd{image}\hlstd{(Zt)}
\end{alltt}
\end{kframe}
\includegraphics[width=\maxwidth]{figure/viewZt-1} 

\end{knitrout}
Then we simulate the response vector, weights, and offset.
\begin{knitrout}
\definecolor{shadecolor}{rgb}{0.969, 0.969, 0.969}\color{fgcolor}\begin{kframe}
\begin{alltt}
\hlstd{eta} \hlkwb{<-} \hlkwd{as.numeric}\hlstd{(X} \hlopt \hlkwd{c}\hlstd{(}\hlnum{1}\hlstd{,} \hlnum{1}\hlstd{)} \hlopt{+} \hlkwd{t}\hlstd{(Zt)} \hlopt \hlkwd{t}\hlstd{(Lambdat)} \hlopt \hlkwd{rnorm}\hlstd{(q))}
\hlstd{y} \hlkwb{<-} \hlkwd{rbinom}\hlstd{(n,} \hlnum{1}\hlstd{,} \hlkwd{plogis}\hlstd{(eta))}
\hlstd{weights} \hlkwb{<-} \hlkwd{rep}\hlstd{(}\hlnum{1}\hlstd{, n); offset} \hlkwb{<-} \hlkwd{rep}\hlstd{(}\hlnum{0}\hlstd{, n)}
\end{alltt}
\end{kframe}
\end{knitrout}

\subsubsection{Organize the parameter vector}

To fit this model we need to use a general nonlinear optimizer, which
usually require a single parameter vector.  Therefore, I put all three
types of parameters in a single vector.
\begin{knitrout}
\definecolor{shadecolor}{rgb}{0.969, 0.969, 0.969}\color{fgcolor}\begin{kframe}
\begin{alltt}
\hlstd{initPars} \hlkwb{<-} \hlkwd{c}\hlstd{(}\hlkwc{covar} \hlstd{=} \hlnum{1}\hlstd{,}
              \hlkwc{fixef} \hlstd{=} \hlkwd{c}\hlstd{(}\hlnum{0}\hlstd{,} \hlnum{0}\hlstd{),}
              \hlkwc{loads} \hlstd{=} \hlkwd{rnorm}\hlstd{(}\hlnum{10}\hlstd{))}
\end{alltt}
\end{kframe}
\end{knitrout}
Importantly, the deviance function needs to know how to find the
different types of parameters, and one specifies this with a list of
indices.
\begin{knitrout}
\definecolor{shadecolor}{rgb}{0.969, 0.969, 0.969}\color{fgcolor}\begin{kframe}
\begin{alltt}
\hlstd{parInds} \hlkwb{<-} \hlkwd{list}\hlstd{(}\hlkwc{covar} \hlstd{=} \hlnum{1}\hlstd{,}
                \hlkwc{fixef} \hlstd{=} \hlnum{2}\hlopt{:}\hlnum{3}\hlstd{,}
                \hlkwc{loads} \hlstd{=} \hlnum{4}\hlopt{:}\hlnum{13}\hlstd{)}
\end{alltt}
\end{kframe}
\end{knitrout}
Because this is a general approach, we must specify functions that
take the parameter vectors and update the various objects.  In
particular, we need a function to map the \texttt{covar} parameters
into the nonzero values of \texttt{Lambdat} (stored in
\texttt{Lambdat@x}).  Similarly, we need to map the factor loadings,
\texttt{loads}, into the nonzero values of \texttt{Zt}.
\begin{knitrout}
\definecolor{shadecolor}{rgb}{0.969, 0.969, 0.969}\color{fgcolor}\begin{kframe}
\begin{alltt}
\hlstd{mapToCovFact} \hlkwb{<-} \hlkwa{function}\hlstd{(}\hlkwc{covar}\hlstd{) covar} \hlopt{*} \hlstd{covTemplate}\hlopt{@}\hlkwc{x}
\hlstd{mapToModMat} \hlkwb{<-} \hlkwa{function}\hlstd{(}\hlkwc{loads}\hlstd{)} \hlkwd{rep}\hlstd{(loads,} \hlnum{18}\hlstd{)}
\end{alltt}
\end{kframe}
\end{knitrout}
Note that the covariance factor is updated as though it were a
Brownian motion model with parameter \texttt{covar} controlling the
rate of evolution.  Here I use the covariance factor for introducing
phylogenetic information, but the random effects model matrix could
also be used.  In this case however, I wanted to illustrate the
possibility of including factor loadings, which should be in the model
matrix.  In this illustrative example, the parameters have an effect
on every nonzero element of the matrices.  However, often (usually)
there are no factor loadings and in this case, \texttt{mapToModMat}
should just return the same nonzero values for any value of the
loadings.  Here is an example of using the mapping functions to update
the two sparse matrices.
\begin{knitrout}
\definecolor{shadecolor}{rgb}{0.969, 0.969, 0.969}\color{fgcolor}\begin{kframe}
\begin{alltt}
\hlstd{Lambdat}\hlopt{@}\hlkwc{x} \hlkwb{<-} \hlkwd{mapToCovFact}\hlstd{(initPars[parInds}\hlopt{$}\hlstd{covar])}
\hlstd{Zt}\hlopt{@}\hlkwc{x} \hlkwb{<-} \hlkwd{mapToModMat}\hlstd{(initPars[parInds}\hlopt{$}\hlstd{loads])}
\end{alltt}
\end{kframe}
\end{knitrout}

\subsubsection{Construct the deviance function}

The interesting thing to report here is that once all of these
structures and mappings are produced, the computation of the deviance
function is now quite straightforward.
\begin{knitrout}
\definecolor{shadecolor}{rgb}{0.969, 0.969, 0.969}\color{fgcolor}\begin{kframe}
\begin{alltt}
\hlstd{devfun} \hlkwb{<-} \hlkwd{mkGeneralGlmerDevfun}\hlstd{(y, X, Zt, Lambdat,}
                               \hlstd{weights, offset,}
                               \hlstd{initPars, parInds,}
                               \hlstd{mapToCovFact, mapToModMat)}
\end{alltt}
\end{kframe}
\end{knitrout}
Here is an example of evaluating it.
\begin{knitrout}
\definecolor{shadecolor}{rgb}{0.969, 0.969, 0.969}\color{fgcolor}\begin{kframe}
\begin{alltt}
\hlkwd{devfun}\hlstd{(initPars)}
\end{alltt}
\begin{verbatim}
## [1] 81.67293
\end{verbatim}
\end{kframe}
\end{knitrout}

\subsubsection{Optimize the deviance function}

We may now use any nonlinear optimizer.
\begin{knitrout}
\definecolor{shadecolor}{rgb}{0.969, 0.969, 0.969}\color{fgcolor}\begin{kframe}
\begin{alltt}
\hlstd{opt} \hlkwb{<-} \hlstd{minqa}\hlopt{:::}\hlkwd{bobyqa}\hlstd{(initPars, devfun)}
\end{alltt}
\end{kframe}
\end{knitrout}
And here are the optimum parameter values.
\begin{knitrout}
\definecolor{shadecolor}{rgb}{0.969, 0.969, 0.969}\color{fgcolor}\begin{kframe}
\begin{alltt}
\hlkwd{setNames}\hlstd{(opt}\hlopt{$}\hlstd{par,} \hlkwd{names}\hlstd{(initPars))}
\end{alltt}
\begin{verbatim}
##      covar     fixef1     fixef2     loads1     loads2     loads3 
##  0.6838425  1.1446008  1.4815679 -0.2306627  0.4113720 -0.7232882 
##     loads4     loads5     loads6     loads7     loads8     loads9 
## -2.7754969  0.5963942  0.8395375 -0.2539937  1.6463548 -0.2929479 
##    loads10 
##  0.5309963
\end{verbatim}
\end{kframe}
\end{knitrout}

\subsection{A latent factor model}

\begin{knitrout}
\definecolor{shadecolor}{rgb}{0.969, 0.969, 0.969}\color{fgcolor}\begin{kframe}
\begin{alltt}
\hlkwd{library}\hlstd{(reo)}
\hlkwd{library}\hlstd{(multitable)}
\end{alltt}
\end{kframe}
\end{knitrout}

Here is an example with Don Jackson's Masters thesis.
\begin{knitrout}
\definecolor{shadecolor}{rgb}{0.969, 0.969, 0.969}\color{fgcolor}\begin{kframe}
\begin{alltt}
\hlkwd{data}\hlstd{(fish)}
\hlkwd{data}\hlstd{(limn)}
\hlstd{Y} \hlkwb{<-} \hlkwd{as.matrix}\hlstd{(fish)}
\hlstd{n} \hlkwb{<-} \hlkwd{nrow}\hlstd{(Y)}
\hlstd{m} \hlkwb{<-} \hlkwd{ncol}\hlstd{(Y)}
\hlstd{x} \hlkwb{<-} \hlkwd{as.vector}\hlstd{(}\hlkwd{scale}\hlstd{(limn}\hlopt{$}\hlstd{pH))}
\hlstd{dl} \hlkwb{<-} \hlkwd{data.list}\hlstd{(}\hlkwc{Y} \hlstd{= Y,} \hlkwc{x} \hlstd{= x,}
                \hlkwc{dimids} \hlstd{=} \hlkwd{c}\hlstd{(}\hlstr{"sites"}\hlstd{,} \hlstr{"species"}\hlstd{))}
\hlstd{dl} \hlkwb{<-} \hlkwd{dims_to_vars}\hlstd{(dl)}
\hlkwd{summary}\hlstd{(dl)}
\end{alltt}
\begin{verbatim}
##            Y     x sites species
## sites   TRUE  TRUE  TRUE   FALSE
## species TRUE FALSE FALSE    TRUE
\end{verbatim}
\end{kframe}
\end{knitrout}
I use the \texttt{multitable} package to organize multivariate data
(could put traits in there too).  Convert it to long format,
\begin{knitrout}
\definecolor{shadecolor}{rgb}{0.969, 0.969, 0.969}\color{fgcolor}\begin{kframe}
\begin{alltt}
\hlkwd{head}\hlstd{(df} \hlkwb{<-} \hlkwd{as.data.frame}\hlstd{(dl))}
\end{alltt}
\begin{verbatim}
##                Y          x       sites species
## 3 Island.PS    0  0.1843841    3 Island      PS
## Austin.PS      0 -0.8325221      Austin      PS
## Bear.PS        1  0.8277329        Bear      PS
## Bentshoe.PS    1 -0.8532753    Bentshoe      PS
## Big East.PS    1 -0.6249903    Big East      PS
## Big Orillia.PS 1  0.2051373 Big Orillia      PS
\end{verbatim}
\end{kframe}
\end{knitrout}

The response vector.
\begin{knitrout}
\definecolor{shadecolor}{rgb}{0.969, 0.969, 0.969}\color{fgcolor}\begin{kframe}
\begin{alltt}
\hlstd{y} \hlkwb{<-} \hlstd{df}\hlopt{$}\hlstd{Y}
\hlstd{weights} \hlkwb{<-} \hlkwd{rep}\hlstd{(}\hlnum{1}\hlstd{,} \hlkwd{length}\hlstd{(y)); offset} \hlkwb{<-} \hlkwd{rep}\hlstd{(}\hlnum{0}\hlstd{,} \hlkwd{length}\hlstd{(y))}
\end{alltt}
\end{kframe}
\end{knitrout}
The fixed effects design matrix.
\begin{knitrout}
\definecolor{shadecolor}{rgb}{0.969, 0.969, 0.969}\color{fgcolor}\begin{kframe}
\begin{alltt}
\hlstd{X} \hlkwb{<-} \hlkwd{model.matrix}\hlstd{(Y} \hlopt{~} \hlstd{x, df)[,]}
\end{alltt}
\end{kframe}
\end{knitrout}
The random effects design matrix.
\begin{knitrout}
\definecolor{shadecolor}{rgb}{0.969, 0.969, 0.969}\color{fgcolor}\begin{kframe}
\begin{alltt}
\hlstd{Jspecies} \hlkwb{<-} \hlkwd{t}\hlstd{(}\hlkwd{as}\hlstd{(}\hlkwd{as.factor}\hlstd{(df}\hlopt{$}\hlstd{species),} \hlkwc{Class} \hlstd{=} \hlstr{"sparseMatrix"}\hlstd{))}
\hlstd{Zt} \hlkwb{<-} \hlkwd{KhatriRao}\hlstd{(}\hlkwd{t}\hlstd{(Jspecies),} \hlkwd{t}\hlstd{(X))}
\end{alltt}
\end{kframe}
\end{knitrout}
Now this design matrix only contains `traditional' random effects, not
factor loadings.
\begin{knitrout}
\definecolor{shadecolor}{rgb}{0.969, 0.969, 0.969}\color{fgcolor}\begin{kframe}
\begin{alltt}
\hlstd{nFreeLoadings} \hlkwb{<-} \hlstd{m}
\hlstd{U} \hlkwb{<-} \hlkwd{matrix}\hlstd{(}\hlnum{1}\hlopt{:}\hlstd{nFreeLoadings,} \hlkwc{nrow} \hlstd{= m,} \hlkwc{ncol} \hlstd{=} \hlnum{1}\hlstd{)}
\hlstd{latentVarNames} \hlkwb{<-} \hlstr{"latent"}
\hlstd{U} \hlkwb{<-} \hlkwd{setNames}\hlstd{(}\hlkwd{as.data.frame}\hlstd{(U), latentVarNames)}
\hlstd{latentData} \hlkwb{<-} \hlkwd{data.list}\hlstd{(U,} \hlkwc{drop} \hlstd{=} \hlnum{FALSE}\hlstd{,} \hlkwc{dimids} \hlstd{=} \hlstr{"species"}\hlstd{)}
\hlstd{df} \hlkwb{<-} \hlkwd{as.data.frame}\hlstd{(dl} \hlopt{+} \hlstd{latentData)}
\hlstd{Jsites} \hlkwb{<-} \hlkwd{with}\hlstd{(df,} \hlkwd{t}\hlstd{(}\hlkwd{as}\hlstd{(}\hlkwd{as.factor}\hlstd{(sites),} \hlkwc{Class} \hlstd{=} \hlstr{"sparseMatrix"}\hlstd{)))}
\hlstd{Zt} \hlkwb{<-} \hlkwd{rBind}\hlstd{(}\hlkwd{KhatriRao}\hlstd{(}\hlkwd{t}\hlstd{(Jsites),} \hlkwd{t}\hlstd{(}\hlkwd{model.matrix}\hlstd{(Y} \hlopt{~} \hlnum{0} \hlopt{+} \hlstd{latent, df))), Zt)}
\end{alltt}
\end{kframe}
\end{knitrout}

\begin{knitrout}
\definecolor{shadecolor}{rgb}{0.969, 0.969, 0.969}\color{fgcolor}\begin{kframe}
\begin{alltt}
\hlstd{templateFact} \hlkwb{<-} \hlkwd{sparseMatrix}\hlstd{(}\hlkwc{i} \hlstd{=} \hlnum{1}\hlopt{:}\hlstd{n,} \hlkwc{j} \hlstd{=} \hlnum{1}\hlopt{:}\hlstd{n,} \hlkwc{x} \hlstd{=} \hlkwd{rep}\hlstd{(}\hlnum{1}\hlstd{, n))}
\hlstd{ii} \hlkwb{<-} \hlkwd{rep}\hlstd{(}\hlnum{1}\hlopt{:}\hlnum{2}\hlstd{,} \hlnum{1}\hlopt{:}\hlnum{2}\hlstd{); jj} \hlkwb{<-} \hlkwd{sequence}\hlstd{(}\hlnum{1}\hlopt{:}\hlnum{2}\hlstd{)}
\hlstd{templateRanef} \hlkwb{<-} \hlkwd{sparseMatrix}\hlstd{(}\hlkwc{i} \hlstd{= ii,} \hlkwc{j} \hlstd{= jj,} \hlkwc{x} \hlstd{=} \hlnum{1} \hlopt{*} \hlstd{(ii} \hlopt{==} \hlstd{jj))}
\hlstd{Lambdat} \hlkwb{<-} \hlkwd{.bdiag}\hlstd{(}\hlkwd{c}\hlstd{(}\hlkwd{list}\hlstd{(templateFact),}
                    \hlkwd{rep}\hlstd{(}\hlkwd{list}\hlstd{(templateRanef), m)))}
\end{alltt}
\end{kframe}
\end{knitrout}

\begin{knitrout}
\definecolor{shadecolor}{rgb}{0.969, 0.969, 0.969}\color{fgcolor}\begin{kframe}
\begin{alltt}
\hlstd{mapToModMat} \hlkwb{<-} \hlkwd{local}\hlstd{(\{}
    \hlstd{Ztx} \hlkwb{<-} \hlstd{Zt}\hlopt{@}\hlkwc{x}
    \hlstd{Zwhich} \hlkwb{<-} \hlstd{Zt}\hlopt{@}\hlkwc{i} \hlopt \hlstd{(}\hlkwd{seq_len}\hlstd{(n)} \hlopt{-} \hlnum{1}\hlstd{)}
    \hlstd{Zind} \hlkwb{<-} \hlstd{Zt}\hlopt{@}\hlkwc{x}\hlstd{[Zwhich]}
    \hlstd{loadInd} \hlkwb{<-} \hlnum{1}\hlopt{:}\hlstd{nFreeLoadings}
    \hlkwa{function}\hlstd{(}\hlkwc{loads}\hlstd{) \{}
        \hlstd{Ztx[Zwhich]} \hlkwb{<-} \hlstd{loads[Zind]}
        \hlkwd{return}\hlstd{(Ztx)}
    \hlstd{\}}
\hlstd{\})}
\end{alltt}
\end{kframe}
\end{knitrout}

\begin{knitrout}
\definecolor{shadecolor}{rgb}{0.969, 0.969, 0.969}\color{fgcolor}\begin{kframe}
\begin{alltt}
\hlstd{mapToCovFact} \hlkwb{<-} \hlkwd{local}\hlstd{(\{}
    \hlstd{Lambdatx} \hlkwb{<-} \hlstd{Lambdat}\hlopt{@}\hlkwc{x}
    \hlstd{LambdaWhich} \hlkwb{<-} \hlstd{(n}\hlopt{+}\hlnum{1}\hlstd{)}\hlopt{:}\hlkwd{length}\hlstd{(Lambdatx)}
    \hlstd{LindTemplate} \hlkwb{<-} \hlstd{ii} \hlopt{+} \hlnum{2} \hlopt{*} \hlstd{(jj} \hlopt{-} \hlnum{1}\hlstd{)} \hlopt{-} \hlkwd{choose}\hlstd{(jj,} \hlnum{2}\hlstd{)}
    \hlstd{Lind} \hlkwb{<-} \hlkwd{rep}\hlstd{(LindTemplate, m)}
    \hlkwa{function}\hlstd{(}\hlkwc{covar}\hlstd{) \{}
        \hlstd{Lambdatx[LambdaWhich]} \hlkwb{<-} \hlstd{covar[Lind]}
        \hlkwd{return}\hlstd{(Lambdatx)}
    \hlstd{\}}
\hlstd{\})}
\end{alltt}
\end{kframe}
\end{knitrout}

\begin{knitrout}
\definecolor{shadecolor}{rgb}{0.969, 0.969, 0.969}\color{fgcolor}\begin{kframe}
\begin{alltt}
\hlstd{initPars} \hlkwb{<-} \hlkwd{c}\hlstd{(}\hlkwc{covar} \hlstd{=} \hlkwd{c}\hlstd{(}\hlnum{1}\hlstd{,} \hlnum{0}\hlstd{,} \hlnum{1}\hlstd{),}
              \hlkwc{fixef} \hlstd{=} \hlkwd{c}\hlstd{(}\hlnum{0}\hlstd{,} \hlnum{0}\hlstd{),}
              \hlkwc{loads} \hlstd{=} \hlkwd{rep}\hlstd{(}\hlnum{0}\hlstd{, m))}
\hlstd{parInds} \hlkwb{<-} \hlkwd{list}\hlstd{(}\hlkwc{covar} \hlstd{=} \hlnum{1}\hlopt{:}\hlnum{3}\hlstd{,}
                \hlkwc{fixef} \hlstd{=} \hlnum{4}\hlopt{:}\hlnum{5}\hlstd{,}
                \hlkwc{loads} \hlstd{= (}\hlnum{1}\hlopt{:}\hlstd{m)}\hlopt{+}\hlnum{5}\hlstd{)}
\end{alltt}
\end{kframe}
\end{knitrout}

\begin{knitrout}
\definecolor{shadecolor}{rgb}{0.969, 0.969, 0.969}\color{fgcolor}\begin{kframe}
\begin{alltt}
\hlstd{devfun} \hlkwb{<-} \hlkwd{mkGeneralGlmerDevfun}\hlstd{(y, X, Zt, Lambdat,}
                               \hlstd{weights, offset,}
                               \hlstd{initPars, parInds,}
                               \hlstd{mapToCovFact, mapToModMat)}
\end{alltt}
\end{kframe}
\end{knitrout}

\begin{knitrout}
\definecolor{shadecolor}{rgb}{0.969, 0.969, 0.969}\color{fgcolor}\begin{kframe}
\begin{alltt}
\hlstd{opt} \hlkwb{<-} \hlstd{minqa}\hlopt{:::}\hlkwd{bobyqa}\hlstd{(initPars, devfun)}
\end{alltt}


{\ttfamily\noindent\color{warningcolor}{\#\# Warning in commonArgs(par, fn, control, environment()): maxfun < 10 * length(par)\textasciicircum{}2 is not recommended.}}\begin{alltt}
\hlkwd{setNames}\hlstd{(opt}\hlopt{$}\hlstd{par,} \hlkwd{names}\hlstd{(initPars))}
\end{alltt}
\begin{verbatim}
##      covar1      covar2      covar3      fixef1      fixef2      loads1 
##  1.96405248 -1.47457951  0.56376024 -2.26248775  0.81707347 -0.29383447 
##      loads2      loads3      loads4      loads5      loads6      loads7 
##  0.16413041 -0.39519344 -4.90990508 -1.32109526 -0.91997268 -0.98342366 
##      loads8      loads9     loads10     loads11     loads12     loads13 
##  0.11042682 -0.29120583 -0.73886605 -0.96269062 -0.53656420 -1.32888318 
##     loads14     loads15     loads16     loads17     loads18     loads19 
##  3.00980086 -0.36394912 -3.47314050 -0.22687037 -2.54996759 -1.28776228 
##     loads20     loads21     loads22     loads23     loads24     loads25 
## -0.35752948 -3.01398890  3.01361676 -1.22801409 -2.63378068  0.10737366 
##     loads26     loads27     loads28     loads29     loads30 
## -0.09197036  1.02311855 -0.20881364 -2.17348057 -2.17346329
\end{verbatim}
\end{kframe}
\end{knitrout}

\begin{knitrout}
\definecolor{shadecolor}{rgb}{0.969, 0.969, 0.969}\color{fgcolor}\begin{kframe}
\begin{alltt}
\hlstd{siteScores} \hlkwb{<-} \hlkwd{cbind}\hlstd{(}\hlkwc{pH} \hlstd{= x,}
                    \hlkwc{latent} \hlstd{=} \hlkwd{environment}\hlstd{(devfun)}\hlopt{$}\hlstd{pp}\hlopt{$}\hlkwd{b}\hlstd{(}\hlnum{1}\hlstd{)[}\hlnum{1}\hlopt{:}\hlstd{n])}
\hlstd{speciesScores} \hlkwb{<-} \hlkwd{cbind}\hlstd{(}\hlkwc{pH} \hlstd{=} \hlkwd{environment}\hlstd{(devfun)}\hlopt{$}\hlstd{pp}\hlopt{$}\hlkwd{b}\hlstd{(}\hlnum{1}\hlstd{)[}\hlopt{-}\hlstd{(}\hlnum{1}\hlopt{:}\hlstd{(n}\hlopt{+}\hlstd{m))],}
                       \hlkwc{latent} \hlstd{= opt}\hlopt{$}\hlstd{par[parInds}\hlopt{$}\hlstd{loads])}
\hlkwd{rownames}\hlstd{(siteScores)} \hlkwb{<-} \hlkwd{rownames}\hlstd{(Y)}
\hlkwd{rownames}\hlstd{(speciesScores)} \hlkwb{<-} \hlkwd{colnames}\hlstd{(Y)}
\hlkwd{biplot}\hlstd{(siteScores, speciesScores,} \hlkwc{cex} \hlstd{=} \hlnum{0.5}\hlstd{)}
\end{alltt}
\end{kframe}
\includegraphics[width=\maxwidth]{figure/ordination_diagram-1} 

\end{knitrout}

\subsection{A simple phylogenetic example}

Working on example in the `pglmer` helpfile.  Going well ...


\section{Discussion}

An important point is that \texttt{mkGeneralGlmerDevfun} deals with
model fitting only.  However, in any application a model specification
module and model presentation/inference module must also be provided.
At this point, I have spent way way way too too too long thinking
about how to do this in a general way.  I need to get on with
community ecology, and so the way forward is to have this general
approach to model fitting and then treat specification and
presentation separately.

As always, lot's to worry about.
\begin{itemize}
\item I'm expecting lots of bugs at this point
\item Starting values?
\item Control parameters?  Can \texttt{glmerControl} be leveraged?
  Probably.
\end{itemize}

\end{document}
