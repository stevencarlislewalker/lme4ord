\documentclass{article}
\usepackage{amsmath}
\usepackage{bm}
\usepackage{fullpage}

\newcommand{\mat}{\bm}
\newcommand{\trans}{^\top}
\newcommand{\vc}{\mathrm{vec}}
\newcommand{\dnorm}{\mathcal{N}}

\title{lme4ord model}
\author{Steve Walker}
\date{}

\begin{document}

\maketitle

% \begin{equation}
%   \label{eq:1}
%   \mat Z = \begin{bmatrix}
%     \mat U\trans\otimes \mat I_n \\
%     \mat 1_m\trans\otimes\mat I_n \\
%     \mat I_m \otimes\mat 1_n\trans \\
%   \end{bmatrix}\trans
% \end{equation}

% \begin{equation}
%   \label{eq:2}
%   \mat Y = \mat X\mat\beta + \mat b_n\mat 1_m\trans + \mat 1_n\mat b_m\trans + \mat U\mat V\trans
% \end{equation}

\section{The basic model}

The matrix-valued linear predictor has entries given by,
\begin{equation}
  \label{eq:3}
  \eta_{ij} = \sum_{k = 1}^p x_{ik}\beta_{k} + b_{row, i} + b_{col, j} + \sum_{l = 1}^d u_{il}v_{jl}
\end{equation}
with random parameters,
\begin{equation}
  \label{eq:4}
  b_{row, i} \sim \dnorm(0, \theta_{row})
\end{equation}
\begin{equation}
  \label{eq:4}
  b_{col, j} \sim \dnorm(0, \theta_{col})
\end{equation}
\begin{equation}
  \label{eq:5}
  v_{jl} \sim \dnorm(0, \theta_{axes})
\end{equation}
and where the $x$'s are known and the $\beta$'s and $u$'s are constant parameters.

\section{The trouble}

There have to be constraints on the $u$'s to keep the model identifiable. 

\end{document}
