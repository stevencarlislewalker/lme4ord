\documentclass{article}\usepackage[]{graphicx}\usepackage[]{color}
%% maxwidth is the original width if it is less than linewidth
%% otherwise use linewidth (to make sure the graphics do not exceed the margin)
\makeatletter
\def\maxwidth{ %
  \ifdim\Gin@nat@width>\linewidth
    \linewidth
  \else
    \Gin@nat@width
  \fi
}
\makeatother

\definecolor{fgcolor}{rgb}{0.345, 0.345, 0.345}
\newcommand{\hlnum}[1]{\textcolor[rgb]{0.686,0.059,0.569}{#1}}%
\newcommand{\hlstr}[1]{\textcolor[rgb]{0.192,0.494,0.8}{#1}}%
\newcommand{\hlcom}[1]{\textcolor[rgb]{0.678,0.584,0.686}{\textit{#1}}}%
\newcommand{\hlopt}[1]{\textcolor[rgb]{0,0,0}{#1}}%
\newcommand{\hlstd}[1]{\textcolor[rgb]{0.345,0.345,0.345}{#1}}%
\newcommand{\hlkwa}[1]{\textcolor[rgb]{0.161,0.373,0.58}{\textbf{#1}}}%
\newcommand{\hlkwb}[1]{\textcolor[rgb]{0.69,0.353,0.396}{#1}}%
\newcommand{\hlkwc}[1]{\textcolor[rgb]{0.333,0.667,0.333}{#1}}%
\newcommand{\hlkwd}[1]{\textcolor[rgb]{0.737,0.353,0.396}{\textbf{#1}}}%

\usepackage{framed}
\makeatletter
\newenvironment{kframe}{%
 \def\at@end@of@kframe{}%
 \ifinner\ifhmode%
  \def\at@end@of@kframe{\end{minipage}}%
  \begin{minipage}{\columnwidth}%
 \fi\fi%
 \def\FrameCommand##1{\hskip\@totalleftmargin \hskip-\fboxsep
 \colorbox{shadecolor}{##1}\hskip-\fboxsep
     % There is no \\@totalrightmargin, so:
     \hskip-\linewidth \hskip-\@totalleftmargin \hskip\columnwidth}%
 \MakeFramed {\advance\hsize-\width
   \@totalleftmargin\z@ \linewidth\hsize
   \@setminipage}}%
 {\par\unskip\endMakeFramed%
 \at@end@of@kframe}
\makeatother

\definecolor{shadecolor}{rgb}{.97, .97, .97}
\definecolor{messagecolor}{rgb}{0, 0, 0}
\definecolor{warningcolor}{rgb}{1, 0, 1}
\definecolor{errorcolor}{rgb}{1, 0, 0}
\newenvironment{knitrout}{}{} % an empty environment to be redefined in TeX

\usepackage{alltt}
\usepackage{amsmath}
\usepackage{bm}

\newcommand{\trans}{^\top}
\newcommand{\link}{\ensuremath{\mathrm{link}}}
\newcommand{\mc}{\mathcal}
\newcommand{\vc}{\ensuremath{\mathrm{vec}}}
\newcommand{\code}[1]{\texttt{#1}}

\title{Random effects term for factor analysis: factAnal}
\author{Steve Walker}
\date{}
\IfFileExists{upquote.sty}{\usepackage{upquote}}{}
\begin{document}

\maketitle

To fit a (possibly unbalanced) factor analysis random effects term
using \code{lme4ord}, one uses the following \code{R} \code{formula}.
\begin{equation}
  \label{eq:6}
  \mathtt{respVar \sim factAnal(0 + obsFac\, | \,varFac)}
\end{equation}
where \code{respVar} is the response vector and \code{obsFac} and
\code{varFac} are \code{R} factors indicating the multivariate
observation and variable associated with each row in the data set.
For example, in ecology \code{codeFac} would normally indicate the
site and \code{varFac} the species associated with each abundance
observation in \code{respVar}.

Let $\bm B$ be an $l$ by $d$ matrix of loadings over $l$ variables
(often species) characterized by $d$ latent axes. All elements above
the diagonal of this loadings matrix are constrained to be zero, all
other elements are free parameters.

Let $\bm
J_{\text{var}}$ and $\bm J_{\text{obs}}$ be an $l$ The random effects
model matrix for the factor analysis term is then given by the
following expression.
\begin{equation}
  \label{eq:5}
  \bm Z\trans = (\bm B\trans\bm J_{\text{var}}\trans) * J_{\text{obs}}\trans
\end{equation}

\newpage

\begin{equation}
  \label{eq:2}
  \link\left(\mc M\right) = 
  \bm X \bm B
\end{equation}
\begin{equation}
  \label{eq:3}
  x_{ij} \sim \mc N(0, 1)
\end{equation}
where $\mc M$ is an $n$ by $m$ matrix giving the mean of the response
matrix, $\bm Y$; $\bm X$ is an $n$ by $d$ matrix of latent factors,
$\bm B$ is a $d$ by $m$ matrix of loadings.  This equation can be
vectorized as,
\begin{equation}
  \label{eq:4}
  \link\left(\vc(\mc M)\right) = 
  \left(\bm B\trans \otimes \bm I_n\right)
  \vc(\bm X)
\end{equation}

\section{Random effects model matrix and relative covariance factor}

Let $\bm X$ be an $l$ by $d$ matrix of loadings over $l$ objects
(often species) characterized by $d$ latent dimensions. Let $\bm J$ be
an $l$ The random
effects model matrix for the factor analysis term is then given by the
following expression.
\begin{equation}
  \label{eq:1}
  \bm Z\trans = 
  \left(
    \bm X\trans \bm J\trans_{\text{obj}}
  \right) * 
  \left(
    \bm J\trans_{\text{var}}
  \right)
\end{equation}
The relative covariance factor is simply an $m$ by $m$ identity
matrix, where $m$ is the number of multivariate observations.

The parameters of $\bm Z\trans$ are in the matrix $\bm X$.  

\end{document}
